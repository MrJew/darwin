CherryPy is a minimalistic python object-oriented framework which focuses on flexibility and has a reliable HTTP/1.1-complient, WSGI thread
pooled web server. It requires a small amount of source code to run it, making it very useful to take the purpose of an \textit{evaluator}. The \textit{evaluator}
needs to be light in order to run either with multiple instances on a single server or run on a small devices like a RaspnerryPi.

\begin{lstlisting}[language=Python,caption={Example CherryPy webservice that has the method "hello"},label={lst:CherryPy}]
class HelloWorld(BaseCherryPy):

    def hello(self):
        return "Hello world!"
    hello.exposed = True

	\end{lstlisting}
The way CherryPy works is it maps each method written in a class by adding it to the server URL name so it can be called. Each of those methods has to have
 an \textit{exposed} flag set to true as shown in \ref{lst:CherryPy} so CherryPy can see it. This class is used by the CherryPy framework which handles
all the URL parsing. For example if we have a method as shown in \ref{lst:CherryPy} and the hostname is \textit{http://localhost} then when accessing the server
with URL \textit{http://localhost/hello} the URL and its parameters if existing will be forwarded to the method \textit{hello}. After that the hello method
will return the contents and CherryPy will format the response in a correct HTTP format.
