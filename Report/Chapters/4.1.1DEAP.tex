DEAP (Distributed Evolutionary Algorithms in Python) is a framework that supports both genetic algorithms and genetic programming.
It seeks to make algorithms explicit and data structures transparent. Hence making genetic programming easy by providing an easy way for primitives
to be created and access to crossover and mutation algorithms. Each of its features can be fine tuned through multiple parameters.
\paragraph{}
Similar to Darwin, DEAP emphasizes on loose coupling in its design, it uses multiple components that provide the necessary functionality
for the framework. The \textit{pset} is a class through which all the configuration related to the primitive set is done. Primitives are defined
through mapping any defined function or method and declaring the amount of arguments needed. Arguments for the genetic program are also defined through the \textit{pset} hence 
separating it from other components so it can be extracted and manipulated individually.
\paragraph{}
Another core component of the DEAP framework is the \textit{toolbox}. This is the component where the genetic programming is configured. Parameters
like maximum depth of an individuals are specified as well as the algorithms for crossover and mutation. The framework gives the flexibility to define your
own mutation and crossover algorithms as well as manipulate their rates and the size of the population and generation. Fitness function is specified and
mapped in the toolbox. After configuration of the primitive set and the genetic
programming parameters is done the population is generated. There is an automatic way of doing it through special DEAP functions however for Darwin
a more complex approach was chosen which will be later explained. The statistics module takes the generations and creates
set of statistics that can be displayed ( e.g. mean of a generation, best and worst individual ).


