Before finalizing the idea of the project, research had to be done so previous related work can be found and compared. The project has a very
specific scope and there aren't many projects working towards a similar goal using genetic programming. The two of the projects were
\textit{Gen-o-fix: an embeddable framework for dynamic adaptive genetic improvement programming}\cite{genofix} and \textit{A Genetic Programming Approach to
Automated Software Repair}\cite{softrepair}.
\paragraph{}
\textit{A Genetic Programming Approach to Automated Software Repair} makes project to repair bugs in off-the-shelf legacy C programs.
To achieve that the framework does three main modifications to the standard GP technique:
\begin{enumerate} 
\item genetic operations are localized to the nodes along the execution path of the negative test case; 
\item high-level statements are represented as single nodes in the program tree;
\item genetic operators use existing code in other parts of the program, so new code does not need to be invented.
\end{enumerate}
 Initially \textit{Darwin} was supposed to include similar functionality as well but it was again later excluded and kept for future development.
\paragraph{}
\textit{Gen-o-fix}explores the concept of  genetic improvement programming (GIP) which goal is to automate software
maintenance. \textit{Gen-o-fix} is a GIP framework which allows a software system hosted on the Java Virtual Machine to be continually improved
(e.g. make better predictions; pass more regression tests; reduce power consumption). Similarly to \textit{Gen-o-fix},
\textit{Darwin} is also focused on creating a user-centric tool rather a research-centric one.
\paragraph{}
Both of the frameworks were focused on post-development processes like bug fixing, code improvement and code maintenance. These ideas are different
from \textit{Darwin}'s one of cloning web services however the concept for creating and modifying existing code through genetic programming is observed
in all three of the project. This made the papers a good source of information and approaches used for the development of \textit{Darwin}.