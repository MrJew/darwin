Individuals are created and generated in the \textit{Populator} class. It is a component of the framework where
the genetic programming is done and handles output. It also contains functionality for communicating with other
web services, either an evaluator or a target web service for cloning.
\paragraph{}
Unlike the \textit{Configuration} class the \textit{Populator} doesn't work with the \textit{pset} and \textit{toolbox} DEAP
components directly. It uses them to generate individuals and apply a mutation or crossover algorithm to it. Instead of using
the standard genetic programming algorithm that DEAP provides by calling a single function, the algorithm had to be implemented 
manually. As seen in \ref{lst:populate} the offspring is initialized by copying it from the initial population. This first population is a
generation of random individuals with depth between 1 and 3. 

\begin{lstlisting}[language=Python,caption={Populate function responsible for generating individuals},label={lst:populate},breaklines=true]

 def populate(self):
    for gen in range(self.configuration.gen):
        self.offspring = self.toolbox.select(self.population, len(self.population))
        self.offspring = map(self.toolbox.clone, self.offspring)
        self.crossover()
        self.mutation()
        invalid_ind = self.nonEvaluated(self.offspring)
        fitnesses = self.toolbox.map(self.evaluate, invalid_ind)
        for ind, fit in zip(invalid_ind, fitnesses):
            ind.fitness.values = fit
        self.population = self.offspring
        self.hof.update(self.population)

    self.outputIndividuals()
\end{lstlisting}

To each generation crossover and mutation functions are applied. However both of the functions are modified so each
individual is evaluated separately. The whole avoiding of DEAP's standard genetic programming function is because evaluation
is not done in the \textit{Populator} but by the \textit{Evaluator}. In order to make use of the decorators the crossover
and mutation function a manual substitution of the generation is done as seen in \ref{lst:populate}.

\begin{lstlisting}[language=Python,caption={Modified mutation method to substitute each parent with its children},label={lst:populate}]

def mutation(self):
    for i in range(len(self.offspring)):
        if random.random() < self.configuration.mut:
            self.offspring[i], = self.toolbox.mutate(self.offspring[i])
            del self.offspring[i].fitness.values
\end{lstlisting}