The XML file that is parsed between the GUI and the framework needed to be readable and editable by users. That meant that the
markup used had to follow standard design so it can be modified by users. After several iteration the XML file took the look of \ref{lst:xmlf}
\begin{lstlisting}[language=XML,caption={XML file used by the framework for configuring the needed parameters},label={lst:xmlf}]
<config>
  <mut>0.2</mut>
  <arguments>
    <arg name="r" number="0">1</arg>
    <arg name="r" number="1">0</arg>
    <arg name="r" number="2">1.5</arg>
    <arg name="r" number="3">5</arg>
    <arg name="r" number="4">15</arg>
    <arg name="r" number="5">13.4</arg>
    <arg name="r" number="6">20.1</arg>
    <arg name="r" number="7">132.2</arg>
  </arguments>
  <pop>300</pop>
  <terminals>
    <terminal>3.14</terminal>
  </terminals>
  <cx>0.8</cx>
  <copyUrl>http://api.wolframalpha.com/v2/query</copyUrl>
  <basicPrimitives>
    <primitive>mul</primitive>
    <primitive>pow</primitive>
  </basicPrimitives>
  <evalUrl>http://localhost:8844</evalUrl>
  <gen>300</gen>
</config>
\end{lstlisting}
This designed provided a backup plan in case GUI implementation failed. However the GUI was created
which meant a parser needed to be implemented which was done with the help of \textit{lxml} library.