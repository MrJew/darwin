Code generation is the main technique for automation used in Darwin framework. It uses
the inspect library to gather the data and code needed for code generation. The main methods
for extraction of code are located in \textit{PrimitiveConfig}. The methods that use the inspect library
are extracting the code of the functions chosen as primitives as seen in fig \ref{lst:inspect}.
The method \textit{functionArgs} extracts the arguments hence the signature from target primitive function
given its name.

\begin{lstlisting}[language=Python,caption={Function extracting the arguments of a primitive based on its name},label={lst:inspect}]
def functionArgs(self,funcName):
    func = getattr(self,funcName)
    args = inspect.getargspec(func)[0]
    del args[0] # removes 'self'
    return args
\end{lstlisting}

The method for converting DEAP individual into source is located in PrimitiveConfig as well. In the \textit{helper} module
there are multiple functions constructing different parts of the individual source as seen in figure \ref{lst:gen}.

\begin{lstlisting}[language=Python,caption={Function in PrimitiveConfig class that gathers source code parts of the individual},label={lst:gen}]

def getSource(self, imports,individual,arguments):
    source = generateImports(imports)
    source+= generateFunctions(self,False)
    source += generateMain(arguments,individual,False)
    source += "print main(**sys.argv)\n"
    return source
\end{lstlisting}

The \textit{getSource} method collects source from three main methods located in the \textit{helper} module.

\begin{enumerate}
\item \textit{generateImports} based on a list of module names is generating the import statements.

\item \textit{generateFunctions} extracts the source and the signature of a primitive by removing the self argument in the method. 
\item \textit{generateMain} creates a main function that accepts as parameters the arguments of an individual. Later the DEAP
individual that is represented as a stack of functions is executed through a lambda expression that yields the result
\item the main method is called using system arguments as a dictionary. The result is printed to the standard output
so it can be forwarded to the original context when executed
\end{enumerate}

The same methods are used when generating a webservice as seen in figure \ref{lst:webs}. The difference is the code is wrapped into a class that is used
by CherryPy meaning the primitive signature and the lambda expression need to include the \textit{self} notation. The webservice has
an index method that is called only by accessing the URL that returns the result of the individual.

\begin{lstlisting}[language=Python,caption={A generated web service that is the result of the Darwin framework},label={lst:webs}]

class ClonedWebService(BaseCherryPy):

    def mul(self,a,b):
        return operator.mul(a,b)

    def main(self,r):
        r = float(r)
        ind = lambda r: self.mul(self.mul(3.14, r), r)
        return ind(r)

    def index(self,r):
        return self.main(r)
    index.exposed = True

cherrypy.quickstart(ClonedWebService())
\end{lstlisting}