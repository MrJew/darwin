\textbf{Multiple Technologies -} among the main issues with the project is the amount of technologies used. This meant each of them
had to be understand so quality development in the area can be done. The framework was related to web services, genetic programming,
code generation, GUI which meant extensive research in each areas so the correct libraries or frameworks can be chosen. The learning
curve was steep for the project however each of the fields became easier to work and understand after the basics were clear.
\paragraph{}
\textbf{Proof of Concept -} before the project reached any results it had a \textit{proof of concept} label. Exploring new fields and
expanding your knowledge in specific fields can be fun but it's hard when not a lot of information is presented. Three times during the
project I had to reach out to \texit{stackoverflow.com} for help related to issues I couldn't find anywhere on the web resolved. The first
time it was related to the way Cherrypy processed it's requests and responses. The second one was focusing on the problem of converting
DEAP individuals into executable code or AST. Later on I had to tackle the problem on my own since such a functionality wasn't introduced to DEAP.
However the developers from DEAP were kind enough to provide support for further questions and the code generation has spawned the possibility
for an open source contribution to the DEAP framework with an extension that converts DEAP individuals to AST's. The third time was related
to code execution in python and the way arguments were passed to a sub processed.