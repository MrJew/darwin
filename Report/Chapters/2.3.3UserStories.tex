As mentioned the project is focusing on a user-centric tool. Which meant user perspective on how
the project should work and look was important. A list of user stories was created for the client
explaining the way he and any potential user was going to interact with the framework. The list
of stories was later iterated until there were set of user stories defining each of the sprints
as seen in figure [ref]
\paragraph{}
\textbf{Sprint one - } was focussed on creating an evaluating web service separated from the client.
That was done with the idea that with later development in the project this will be used for
improving a web service using genetic programming. However this idea was left behind but the functional
decision gave the potential ability to evaluate multiple individuals at the same time using multiple
evaluating services so it was a good base for the project. First sprint included configuration of 
basic genetic programming parameters and using a pre-defined primitive set or a user defined one.
However later on both were implemented. Generating python code out of the best solution was another 
feature for sprint one which was complete in order to create the base for a framework. Since
DEAP was already the choice for the genetic programming framework integration with it was a requirement as well.
\paragraph{}
\textbf{Sprint two - } involved using a URL address to set up an existing web service as an \textit{"Oracle"}
which meant using them as a fitness function to the currently evolving web service. Because the user story was the user interacting with
the system only through a single URL address that meant a lot of automated configuration needed to be done to create
a simple easy to use framework. Unit tests had to be
completed to make sure development was on the right track. 
\paragraph{}
\textbf{Sprint two - } aimed for creating a more interactive and easy way to configure the system. Initially that
meant only creating an XML file parsing all the configuration parameters in the framework. However a GUI for generating
the XML file was added to ease the process. The purpose of the GUI was to ease the configuration without removing the
previously existing methods hence keeping the advanced functionality of the framework.