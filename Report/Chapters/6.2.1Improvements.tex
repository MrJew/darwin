\textbf{Expand beyond symbolic regression} - throughout the project symbolic regression was 
used as a proof of concept so the framework functionality can be showed. However non mathematical
web services are harder to implement using the framework and this can be improved so cloning can be
more automated. This can be done by adding a better method for overriding at the evaluating web service
and tweaking the basic primitives sets by adding primitives for control flows.
\paragraph{}
\textbf{AST for code generation} - currently all the code that is generated is in string form and through
unit tests it was proved that it's working well however using string manipulation rather then manipulating ASTs
is not as robust. A good re-factoring plan is to do all the code generation with ASTs and keep generation from
strings as little as possible. This would also help for creating the base for an open source contribution to
the DEAP framework that adds the functionality for individuals to be converted to ASTs and/or executable python code.
\paragraph{}
\textbf{Asynchronous requests} - among the initial requirements of the project was using a cloud of evaluating
web services to evaluate individuals over the web hence improving the speed especially for more complex
individuals. However that idea is a small project on it's own and it was going to bring a lot of complexity to the project.
Since the core of \textit{Darwin} is completed it's a solid place to start from however for dealing with threads
and asynchronous requests python is not the best language to handle the issues, implementing it in C would be much better