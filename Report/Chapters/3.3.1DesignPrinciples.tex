While iterating through design decisions the idea was to make the GUI as common as possible. In many dialogs
and simple graphical user interfaces a similar layout is encountered - menubar on top, left to right and
top to bottom reading, bottom right buttons for completing the task see figure[picture of guis]. Because
of its purpose and the way its used the GUI looks more as a dialogue window - simple pop up that helps 
generate a file/configuration and its used for a short amount of time. That is has more of the look and feel of
a dialogue and inspirations and examples for it are taken from similar designs.

As seen in the final design of the GUI the most important parameters - target IP address and client IP address,
are located at the top left corner. Other parameters text fields are located from left  to right top to bottom
based on their relevance. That's why bottom left is located the button for generating the XML configuration
file hence finishing the process.

In the first iteration of the design additional features like adding custom primitives and selecting imports
for those custom primitives were added as you can see in figure [pic of previous design]. However after discussion with the client it was later
decided that these features are part of the more advanced functionality of the framework. Being so it meant
it was unnecessary to implement them in the GUI since they were available in the code where they would be 
normally reached and modified from a more experienced user.


