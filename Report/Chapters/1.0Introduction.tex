In recent years web services have become a main method of communicating between applications
and accessing needed information, creating similar to a distributive environment for many developers.
Because of the standardized communication used by the web services they are expected to be used even 
more in the future providing even more commodities for both developers and users. The nature of web
services is a black box system which means we are unfamiliar with its contents and we only know how to
access it and what it returns. 
\paragraph{}
That's where the idea came from, for copying a web services using genetic
programming, which became a proof of concept project that reached very far. The genetic programming 
involves generating programs based on set of functions and then evaluating the quality (correctness)
of the program via a fitness function. To prove that cloning of a web service can be achieved using
genetic programming we used a symbolic regression problem where the genetic program needs to find a 
mathematical expression that yields the same result as a given one from a fitness function. But 
instead of using our own fitness function we are using a real web service hence generating a program
that has the same functionality as the web service. Proving this meant that theoretically by increasing 
the complexity of our function set we can achieve copying of an even more complex web service.
