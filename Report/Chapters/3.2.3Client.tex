Before explaining how the client works, DEAP framework needs to be explained. It is configured through set of method calls
to a \textit{toolbox} this \textit{toolbox} is prior configured by a \textit{primitive set} where all the primitives and 
terminals that the genetic program is going to use are defined. The toolbox give easy API for accessing and controlling the individuals,
generations and population as well as methods that invoke mutation, crossover and evaluation over the population.
\paragraph{}
Knowing how the DEAP framework works, it was later obvious that the same parameters that were needed for DEAP's configuration
were going to be needed for Darwin's as well. Through the \textit{Configuration} class shown in figure ~\ref{fig:firstSprint}
we are able to set all the primitive functions and the fitness function needed for the program. Later on the client decided
that there should be a preset number of primitive sets and the user will choose through the GUI which to enable. However in
the case of an advance user if additional primitives need to be added the class can simply be extended and add them as
normal methods. No other configurations regarding the primitives need to be done since Darwin is handling everything else.
[insert class diagram]
\paragraph{}
In figure ~\ref{fig:firstSprint} you can see that for the configuration of the genetic program we need other parameters which
are not specified in the \textit{Configuration} class. When setting up the framework they can be set manually or configured
by an XML file that carries all the information needed. There are only two parameters that are mandatory and the framework
won't work without them - the address of the evaluating web service and the target service that needs to be cloned. Any other
parameter is automatically set up with a standard value for solving the symbolic regression problem. The class that configures
the framework acts like a data model since all the parameters needed through any other step of the framework is instantiated there.
This data model is later used by the \textit{Populator}.
\paragraph{}
The \textit{Populator} class is basically a wrapper around the DEAP framework. What the class do is it creates the first population
then sends each individual over the network to the evaluating web service. After each individual's fitness value is updated it
repeats the process until the number of generations are reached. Besides executing the actual genetic program the \textit(Populator)
logs the traffic between the services and handles the output of the genetic program - either displaying the best individual or 
generating a web service working with the best individual.

